We propose graph neural network methods that learn from hierarchies of graphs representing a sequence of nested graphs obtained from topological filtration. Treating graphs from the perspective of 1-D simplicial complexes, our filtration is defined as being over 1-cells (graph edges). We obtain a sequence of simplicial complexes, where each complex in the series is subset to the subsequent, from a sorted ordering of 1-cells, thresholded by filtration values prescribed to simplexes through a filter function defined as a learning model trained to infer if edges are homophilous, akin to a class similarity measure between 0-cells (graph nodes). The proposed approach is generalizable to any edge based classification task or graph with weighted edges that afford an ordering. We then present two methods for hierarchical graph learning. The former, achievable for any standard GNN model, performs message passing across subsequent graphs in the graph hierarchy, aggregating node embeddings sequentially moving up the graph filtration sequence. the ladder introduces a novel multi-scale message passing scheme with aggregation performed jointly across graphs in the filtration sequence, with learned attention based aggregation or learned topological filtration of subset graph nodes, effectively learning the degree of contribution graphs within the hierarchy contribute to learned aggregated node embeddings. Our results demonstrate promise for increased accuracy and improvement in training time compared to conventional GNNs while we provide exploratory insights into the effect of class imbalance and heterophily in graph learning. 
%\paragraph{Social Impact} Our methods have the potential to minimize the amount of human effort required to extract semantic structure from scientific images, which may contribute to socially beneficial breakthroughs in medicine, materials science, and more.  As training data comes from human annotations, our methods are subject to reproducing annotators' biases.  A socially impactful future direction is to understand the extent to which topological priors reflect semantic structure, which we have found to vary across datasets, and which may also vary for subpopulations within a dataset. 
%\paragraph{Acknowledgments}
% This work was performed under the auspices of the U.S. Department of Energy by Lawrence
% Livermore National Laboratory under Contract DE-AC52-07NA27344, Lawrence Livermore National Security, LLC. and was supported by the LLNL-LDRD Program under Project No. 21-ERD-012.
