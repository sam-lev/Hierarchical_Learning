
\subsection{Multi-Scale Sequence of Graphs and Graph Hierarchy}
\label{sSec:graph_hier}

To model multiscale topological information, we perform topological filtration of a graph at $P$ levels of filter value thresholds. In contrast to hierarchical graph-level representation learning methods that learn to pool each input graph~\cite{ying2018hierarchical}, this gives us several graph representations of the underlying data to use as input to a graph neural network. Computationally, a larger filter threshold value produces a graph higher in the graph hierarchy with higher granularity and corresponds to a 1-D simplicial complex later in the sequence of nested simplicial complexes in the filtration.  Thus, we obtain a hierarchy of graphs $G_1, \ldots, G_P$ where  $\forall p_i$ for $i \in [1, \ldots, P], G_{p_{i-1}} \subseteq G_{p_i}$: that is, $G_{p_{i-1}}$ is an induced subgraph defined on a subset of the vertices in $G_{p_i}$. %
%More formally, if we have persistence levels $p_i$ and $p_j$, such that $p_j < p_i$, then $\msc(p_i) \subset \msc(p_j)$ and $\iG \subset \jG$ for priors graphs $\iG$ and $\jG$. 
Node neighborhoods thus share this property: for each node $v_i \in \iV \cap \jV$, e.g. $\iNbr{v_i}$ for $v_i \in \iV$ and $\jNbr{v_i}$ for $v_i \in \jV$, we have that $\iNbr{v_i} \subseteq \jNbr{v_i}$. %Due to persistence filtrations resulting in a  sequential hierarchy of complexes and corresponding priors graphs,  
We give all graphs the full node set found at the lowest filtration veluelevel of the hierarchy, but nodes that would not otherwise exist in the graph at a higher filter value level are disconnected.%have isolated vertices without connections to any other vertex belonging to a graph of a higher persistence level.  
